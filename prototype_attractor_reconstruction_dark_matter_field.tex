\documentclass[11pt]{article}
\usepackage{amsmath,amssymb}
\usepackage{fullpage}
\usepackage{hyperref}
\hypersetup{colorlinks=true, urlcolor=blue}

\title{Prototype: Attractor Reconstruction of a Dark-Matter Field}
\author{Laura Lopez}
\date{September 27, 2025}

\begin{document}

\maketitle

\begin{abstract}
We propose a fixed-point (attractor) reconstruction of the projected dark-matter field $\psi(x)$
from combined gravitational lensing and stellar-stream data. The method alternates short gravitational
relaxation steps under the Vlasov--Poisson equations with observational pull-backs, producing a candidate
attractor $\psi^\ast$ whose stability and uniqueness can be directly tested.\footnote{This work was inspired
by Geoffrey Hinton’s public remarks that AI may be especially suited to connecting seemingly unrelated concepts:
\emph{Geoffrey Hinton reveals the surprising truth about AI’s limits and potential}, YouTube (2024).
Link: \url{https://youtu.be/n4IQOBka8bc}}%

\par
If a candidate prior (cold, warm/fuzzy, or nonlocal) both fits the data and successfully cross-predicts
independent observations, it provides a concrete discriminator among dark-matter models.
The framework is offered not as a claim of discovery, but as a testable hypothesis designed to guide
further analysis and simulation.
\end{abstract}

\section{Objective}
Reconstruct the projected dark matter field $\psi(x)$ as a fixed point of an operator $F$ that alternates between:
(i) gravitational relaxation under Vlasov--Poisson, and (ii) pullbacks from data (lensing + stellar streams).
Converged solutions are attractors; stability and uniqueness are testable.

\section{Mathematical Formulation}
We define an energy functional:
\begin{equation}
E[\psi] = \tfrac{1}{2}\langle \delta, K^{-1}\delta \rangle
        + \chi^2_{\text{lens}}[\psi]
        + \chi^2_{\text{stream}}[\psi]
        + \lambda \|\nabla \psi\|^2,
\end{equation}
with $\delta = e^\psi - 1$.
This is equivalent to adopting the common transformation $\psi = \log(1+\delta)$, which guarantees $\delta > -1$
and is widely used in cosmology to Gaussianize the density field, handling both overdense and underdense regions.

The kernel $K$ encodes dark-matter microphysics:
parameterized families spanning CDM (baseline), WDM/fuzzy (cutoff), or DIN/plasma-like non-local kernels.
Attractor condition: $\psi^\ast = F(\psi^\ast)$.
Stability: spectral radius $\rho(J)<1$ of the Jacobian $J$ at $\psi^\ast$.

\section{Assumptions}
We hypothesize that the operator $F$ admits stable fixed points, motivated by analogies to attractor dynamics
in Hopfield networks and the relaxation properties of collisionless N-body systems.
Convergence of $F$ is not guaranteed for arbitrary priors or data; safeguards include adaptive step sizes,
multiple initialization strategies, and explicit convergence diagnostics beyond relative change.

The solution may not be unique given the inverse problem’s ill-posed nature; characterizing the solution manifold,
quantifying Bayesian uncertainty, and exploring multiple attractors are important follow-ups.

\section{Inputs}
\begin{itemize}
\item Weak/strong lensing $\kappa$-maps of well-studied fields (e.g. HST Frontier Fields).
\item One stellar stream line-density profile from Gaia DR3 (e.g. GD-1, Pal 5).
\item Optional: rotation curves or velocity dispersions.
\end{itemize}
Relative weights $w_L$ and $w_S$ between lensing and streams can be adaptively tuned to reflect observational
depth and signal-to-noise.

\section{Operator \texorpdfstring{$F$}{F}}
Each iteration consists of:
\begin{enumerate}
\item \textbf{Gravity step:} compute $\delta = e^\psi - 1$, solve $\nabla^2 \phi = 4\pi G \bar{\rho}\, \delta$.
\item \textbf{Data pullback:} forward lensing projection $\kappa(\psi)$; integrate stellar orbits in $\phi$;
   update misfit gradients.
\item \textbf{Prior/proximal step:}
\[
\psi \leftarrow \arg\min \left\{ \tfrac{1}{2}\langle \delta, K^{-1}\delta\rangle
+ \lambda \|\nabla\psi\|^2 + \tfrac{\alpha}{2}\|\psi - \tilde{\psi}\|^2 \right\}.
\]
\end{enumerate}
Repeat until convergence.

\section{Implementation Sketch (pseudocode)}
\begin{verbatim}
for t in range(T):
    delta = np.exp(psi) - 1
    phi = poisson_solve(delta)
    grad_lens   = lensing_adjoint(lensing_project(psi) - kappa_obs)
    grad_stream = stream_adjoint(stream_line_density(phi), stream_obs_power)
    grad_data = wL*grad_lens + wS*grad_stream
    psi_tilde = psi - eta*grad_data
    psi = proximal_fourier(psi_tilde, K, alpha, lambda_reg)
    if rel_change(psi, prev) < eps: break
\end{verbatim}

\section{Nonlocal Priors}
Candidate priors may include Gaussian kernels, Yukawa-type falloffs, or convolutional operators
derived from N-body simulations. Rather than discrete kernel choices, parameterized families can be tested,
spanning CDM, WDM/fuzzy, and plasma-like nonlocality.
Validation relies on cross-prediction: fit lensing and predict streams, or the reverse.
Priors that succeed in both channels provide evidence for the correct microphysics.

\section{Toy Example}
A minimal synthetic test can illustrate feasibility. For instance,
generate a Gaussian clump $\psi(x)$, apply one iteration of $F$
(gravity step + lensing misfit), and verify convergence toward a stable solution.
A 2D toy model, matching the dimensionality of lensing and stream observables, provides the natural proof-of-concept to demonstrate the operator’s behavior and test stability and uniqueness.

\section{Computational Cost}
Full convergence on high-resolution cosmological fields may be expensive.
Reduced-dimension toy models (1D streams, 2D lensing patches)
allow validation before scaling. FFT Poisson solvers, sparse representations,
and GPU acceleration mitigate costs.

\section{Outputs}
\begin{itemize}
\item Convergence history.
\item Stability: spectral radius $\rho(J)$ at $\psi^\ast$, and optionally Lyapunov exponents.
\item Basin robustness: convergence from random seeds.
\item Cross-prediction between lensing and streams.
\end{itemize}

\section{Why It Matters}
Even null results are informative: convergence under CDM alone is nontrivial.
If a non-local prior improves joint fit and stability, that suggests hidden-layer physics.
If both datasets fit individually but only one cross-predicts, the method
discriminates among dark-matter models.
Given possible non-uniqueness, families of attractors may exist; characterizing these
and quantifying uncertainty represent essential next steps.

\section*{Credit \& Rights}
Concept Note: This hypothesis framework was authored by Laura Lopez, September 2025.\\
Licensed under Creative Commons Attribution (CC-BY 4.0).\\
\url{https://creativecommons.org/licenses/by/4.0/}

\end{document}